\documentclass[]{article}
\usepackage{lmodern}
\usepackage{amssymb,amsmath}
\usepackage{ifxetex,ifluatex}
\usepackage{fixltx2e} % provides \textsubscript
\ifnum 0\ifxetex 1\fi\ifluatex 1\fi=0 % if pdftex
  \usepackage[T1]{fontenc}
  \usepackage[utf8]{inputenc}
\else % if luatex or xelatex
  \ifxetex
    \usepackage{mathspec}
  \else
    \usepackage{fontspec}
  \fi
  \defaultfontfeatures{Ligatures=TeX,Scale=MatchLowercase}
\fi
% use upquote if available, for straight quotes in verbatim environments
\IfFileExists{upquote.sty}{\usepackage{upquote}}{}
% use microtype if available
\IfFileExists{microtype.sty}{%
\usepackage[]{microtype}
\UseMicrotypeSet[protrusion]{basicmath} % disable protrusion for tt fonts
}{}
\PassOptionsToPackage{hyphens}{url} % url is loaded by hyperref
\usepackage[unicode=true]{hyperref}
\hypersetup{
            pdfborder={0 0 0},
            breaklinks=true}
\urlstyle{same}  % don't use monospace font for urls
\usepackage{color}
\usepackage{fancyvrb}
\newcommand{\VerbBar}{|}
\newcommand{\VERB}{\Verb[commandchars=\\\{\}]}
\DefineVerbatimEnvironment{Highlighting}{Verbatim}{commandchars=\\\{\}}
% Add ',fontsize=\small' for more characters per line
\newenvironment{Shaded}{}{}
\newcommand{\KeywordTok}[1]{\textcolor[rgb]{0.00,0.44,0.13}{\textbf{#1}}}
\newcommand{\DataTypeTok}[1]{\textcolor[rgb]{0.56,0.13,0.00}{#1}}
\newcommand{\DecValTok}[1]{\textcolor[rgb]{0.25,0.63,0.44}{#1}}
\newcommand{\BaseNTok}[1]{\textcolor[rgb]{0.25,0.63,0.44}{#1}}
\newcommand{\FloatTok}[1]{\textcolor[rgb]{0.25,0.63,0.44}{#1}}
\newcommand{\ConstantTok}[1]{\textcolor[rgb]{0.53,0.00,0.00}{#1}}
\newcommand{\CharTok}[1]{\textcolor[rgb]{0.25,0.44,0.63}{#1}}
\newcommand{\SpecialCharTok}[1]{\textcolor[rgb]{0.25,0.44,0.63}{#1}}
\newcommand{\StringTok}[1]{\textcolor[rgb]{0.25,0.44,0.63}{#1}}
\newcommand{\VerbatimStringTok}[1]{\textcolor[rgb]{0.25,0.44,0.63}{#1}}
\newcommand{\SpecialStringTok}[1]{\textcolor[rgb]{0.73,0.40,0.53}{#1}}
\newcommand{\ImportTok}[1]{#1}
\newcommand{\CommentTok}[1]{\textcolor[rgb]{0.38,0.63,0.69}{\textit{#1}}}
\newcommand{\DocumentationTok}[1]{\textcolor[rgb]{0.73,0.13,0.13}{\textit{#1}}}
\newcommand{\AnnotationTok}[1]{\textcolor[rgb]{0.38,0.63,0.69}{\textbf{\textit{#1}}}}
\newcommand{\CommentVarTok}[1]{\textcolor[rgb]{0.38,0.63,0.69}{\textbf{\textit{#1}}}}
\newcommand{\OtherTok}[1]{\textcolor[rgb]{0.00,0.44,0.13}{#1}}
\newcommand{\FunctionTok}[1]{\textcolor[rgb]{0.02,0.16,0.49}{#1}}
\newcommand{\VariableTok}[1]{\textcolor[rgb]{0.10,0.09,0.49}{#1}}
\newcommand{\ControlFlowTok}[1]{\textcolor[rgb]{0.00,0.44,0.13}{\textbf{#1}}}
\newcommand{\OperatorTok}[1]{\textcolor[rgb]{0.40,0.40,0.40}{#1}}
\newcommand{\BuiltInTok}[1]{#1}
\newcommand{\ExtensionTok}[1]{#1}
\newcommand{\PreprocessorTok}[1]{\textcolor[rgb]{0.74,0.48,0.00}{#1}}
\newcommand{\AttributeTok}[1]{\textcolor[rgb]{0.49,0.56,0.16}{#1}}
\newcommand{\RegionMarkerTok}[1]{#1}
\newcommand{\InformationTok}[1]{\textcolor[rgb]{0.38,0.63,0.69}{\textbf{\textit{#1}}}}
\newcommand{\WarningTok}[1]{\textcolor[rgb]{0.38,0.63,0.69}{\textbf{\textit{#1}}}}
\newcommand{\AlertTok}[1]{\textcolor[rgb]{1.00,0.00,0.00}{\textbf{#1}}}
\newcommand{\ErrorTok}[1]{\textcolor[rgb]{1.00,0.00,0.00}{\textbf{#1}}}
\newcommand{\NormalTok}[1]{#1}
\IfFileExists{parskip.sty}{%
\usepackage{parskip}
}{% else
\setlength{\parindent}{0pt}
\setlength{\parskip}{6pt plus 2pt minus 1pt}
}
\setlength{\emergencystretch}{3em}  % prevent overfull lines
\providecommand{\tightlist}{%
  \setlength{\itemsep}{0pt}\setlength{\parskip}{0pt}}
\setcounter{secnumdepth}{0}
% Redefines (sub)paragraphs to behave more like sections
\ifx\paragraph\undefined\else
\let\oldparagraph\paragraph
\renewcommand{\paragraph}[1]{\oldparagraph{#1}\mbox{}}
\fi
\ifx\subparagraph\undefined\else
\let\oldsubparagraph\subparagraph
\renewcommand{\subparagraph}[1]{\oldsubparagraph{#1}\mbox{}}
\fi

% set default figure placement to htbp
\makeatletter
\def\fps@figure{htbp}
\makeatother


\date{}

\begin{document}

\section{Natural Language Detection}\label{natural-language-detection}

Welcome to the final project for CSC211. \textbf{Read this document
thoroughly before starting on your project.}

Your task is to implement a machine learning algorithm that detects what
language a given text file is written in. You will do this by employing
a method called \textbf{n-gram frequency analysis}. We will restrict our
analysis to languages that use the english alphabet.

\subsection{Machine Learning}\label{machine-learning}

When we use statistics in computer science, this is called
\textbf{machine learning}. It is a beautiful field yet in its relative
infancy. Using the computational power at our fingertips, we use machine
learning (a.k.a. statistical models) to perform a variety of tasks.
These tasks range from predicting disease spread in a population to
finding out which ads to present to you to best catch your attention.
Voice recognition software like Siri and Cortana use machine learning to
better understand what people are saying. We can use machine learning to
look and pictures of someone's face and guess their age with remarkable
accuracy. The variety of tasks that can be tackled with this approach is
limited only by your imagination.

Any machine learning project is implemented with a 2-step process. The
first step is to \textbf{train} an algorithm on a provided
\textbf{sample} of data. In our case, you will be provided with some
text files in various languages. Your algorithm will train on those
files. The second step is to test your \textbf{trained algorithm} on
some files that contain text different from that in the training files.
Done correctly, your algorithm will predict, with remarkable accuracy,
the language that the test file is written in.

\subsection{n-gram Frequency Analysis}\label{n-gram-frequency-analysis}

An \textbf{n-gram} is a sequence of n consecutive characters from a
given sample of text. For this project, you will take \textbf{n = 3}.
From here on out, we will call a sequence of 3 consecutive characters a
\textbf{trigram}.

The idea behind \textbf{trigram frequency analysis} is that you measure
the \textbf{frequency} of each trigram in a given text file. You will
count how many times each trigram occurs in a given text file and you
will store these frequencies in a vector. We will call such vectors
\textbf{frequency vectors}.

You will calculate a frequency vector for each of your training
languages. You will then compute the frequency vector for the test
language. You will then calculate the \textbf{similarity} between the
test frequency vector and each of the training frequency vectors. The
training frequency vector that yields the highest similarity will
correspond to the best matching language.

\subsection{Project Structure and
Deadlines}\label{project-structure-and-deadlines}

This project will be split into two milestones.

\subsection{Milestone 1}\label{milestone-1}

\paragraph{\texorpdfstring{Due: Thursday April 19$^{th}$, 2018 at
11pm}{Due: Thursday April 19$^{th}$, 2018 at 11pm}}\label{due-thursday-april-19th-2018-at-11pm}

For Milestone 1, you will write an algorithm to compute the frequency
vector from a given text file.

\paragraph{Trigrams}\label{trigrams}

As stated earlier, a \textbf{trigram} is a sequence of three consecutive
characters from a given string. For example, given the string ``hello'',
the trigrams are ``hel'', ``ell'', and ``llo''. When calculating what
the trigrams are, you will consider upper and lower case letters to be
the same. You will also ignore any characters that are \textbf{not
letters}. For example, given the string ``This is I.'', the trigrams are
``thi'', ``his'', ``isi'', ``sis'', and ``isi''.

\paragraph{Trigram Frequency Vectors}\label{trigram-frequency-vectors}

In order to compute the similarity between two documents, it is
necessary to represent the trigram frequencies as vectors in the
mathematical sense. Specifically, these vectors need to have the
\textbf{same number of elements} (i.e.~they need to have the same
dimension) and \textbf{those elements need to be in the same order}
across different vectors. While we want you to have great freedom of
design in this project, we must also strive for correctness. As such, we
must standardize the ordering for trigrams.

In class, we discussed an algorithm for representing any trigram of
ASCII characters as an element in a vector \(256^3\) elements long,
using base-256 encoding. However, since we are only concerned with
characters in the english alphabet, we can get away with base-26
encoding. (Remember that we are ignoring any characters that are not
letters, and are treating upper-case letters as if they were lower-case
letters.) The unique charracters under consideration are \textbf{a-z}.
Therefore, we can use vectors that are only \(26^3\) long. Thus, the
encoding you must use is that a maps to 0, b maps to 1, c maps to 2,
\ldots{} , z maps to 26. With this encoding, the ,trigram ``aaa'' would
map to 0 and the trigram ``zzz'' would map to 17,575 (which is \(26^3\)
- 1). As a more useful example, consider the string ``This is I.'':

\begin{itemize}
\tightlist
\item
  The trigram ``thi'' maps to the index \(19*26^2 + 7*26 + 8 = 13,034\).
  Its frequency is 1.
\item
  The trigram ``his'' maps to the index \(7*26^2 + 8*26 + 18 = 4,958\).
  Its frequency is 1.
\item
  The trigram ``isi'' maps to the index \(8*26^2 + 18*26 + 8 = 5,884\).
  Its frequency is 2.
\item
  The trigram ``sis'' maps to the index
  \(18*26^2 + 8*26 + 18 = 12,394\). Its frequency is 1.
\item
  All other indeces in the frequency vector would contain 0.
\end{itemize}

\textbf{Important note:} the leftmost letter in the most significant
digit in our base-26 system.

\paragraph{Milestone requirements}\label{milestone-requirements}

Your program will receive a single command line argument that is the
name of a text file. Your program should start by reading this text
file. It will then calculate what the trigrams are and will store the
frequency of eacy trigram in some way. How you go about storing these
frequencies is entirely up to you. You could use C-arrays, or the C\texttt{++}
\textbf{std::vector} class, or your own class, or anything else that
makes sense.

You must name your output file \textbf{frequencies.o}.

\textbf{Sample command-line input:}

\texttt{\$ ./frequencies.o german}

Your program should print out the integer values of the frequencies in
the order discussed in the section above. The integer values need to be
\textbf{space-separated} and there needs to be a \textbf{new-line
character} after the frequency of the very last trigram.

\paragraph{Grading Rubric}\label{grading-rubric}

\begin{itemize}
\tightlist
\item
  Functional Correctness: 70\%
\item
  Design, Representation, and Comments: 30\%
\end{itemize}

\subsection{Milestone 2}\label{milestone-2}

\paragraph{\texorpdfstring{Due: Monday April 30$^{th}$, 2018 at
11pm}{Due: Monday April 30$^{th}$, 2018 at 11pm}}\label{due-monday-april-30th-2018-at-11pm}

You will calculte a trigram frequency vector for each file in a given
set of training files. You will also calculate the frequency vector for
a test file. You will then compute the similarity between the test
frequency vector and each training frequency vector. You will keep track
of which similarity was the greatest and will output the name of
training file which produced the best match.

\paragraph{Similarity}\label{similarity}

Given two vectors (\textbf{mathematical vectors, not C\texttt{++} vectors}), the
similarity between those vectors can be described using \textbf{cosine
similarity}. This is the cosine of the angle between the two vectors.

A formula for the \textbf{cosine similarity} of two vectors A and B,
where both A and B have n elements, is:

\begin{equation}
\cos^2 \theta = \frac{\Big(\sum\limits_{i=0}^{n-1}{A_i B_i}\Big)^2}{\Big(\sum\limits_{i=0}^{n-1}{A_i^2}\Big) \Big(\sum\limits_{i=0}^{n-1}{B_i^2}\Big)}
\end{equation}

This may look like a scary equation but, when we get down to it, it is
actually pretty simple. Let's deconstruct it and look at its parts.

Let's take a look at the numerator:

\begin{equation}
\sum\limits_{i=0}^{n-1}{A_i B_i}
\end{equation}

This is the \textbf{sum} of the \textbf{element-wise product} of
\textbf{corresponding elements} in the vectors A and B. In math, this is
known as the \textbf{dot-product} of two vectors.

Let's take a look at each term in the denominator:

\begin{equation}
\sum\limits_{i=0}^{n-1}{A_i^2}
\end{equation}

This is the \textbf{square-root} of the \textbf{sum} of the
\textbf{element-wise square} of \textbf{each element} in the vector A.
In math, this is called the \textbf{norm} of a vector. The same goes for
the second term in the denominator.

The value for cosine similarity will range between 0 and 1 inclusive.
The larger this value is between two vectors, the more similar they are.

\paragraph{Pitfalls}\label{pitfalls}

As it so often happens, there are some pitfalls when calculating the
similarity between vectors using built-in C\texttt{++} integer types. The
frequency vectors will have some values that are quite large and when
these values are squared, they \textbf{overlow} even unsigned long long.
Naturally, we are going to have you use \textbf{bigints} to get around
this issue. You will store the intermediate parts of the similarity
calculation as bigints and then use bigint methods to calculate
\(cos^2 \theta\).

Another thing to notice is that similarity is going to be of type
\texttt{double}. We have not implemented bigints that can handle
floating-point types. We will get around this problem by using a little
math trick called \textbf{scaled-division}. The idea is that you: 

\begin{itemize}
\tightlist
\item
  take the numerator and multiply it by some convenient number (e.g. \(1,000,000\)),
\item
  perform bigint division using the \textbf{scaled numerator},
\item
  convert the result into a regular integer,
\item
  and finally perform \textbf{floating-point division} by 1,000,000
\end{itemize}

to get the actual value for \(cos^2 \theta\). You can then calculate the
square-root of this value to get the \textbf{cosine similarity} between
the two vectors.

Note that you will be performing multiplications and divisions with some
really large numbers. The iterative bigint \texttt{multiply} and
\texttt{divide} methods you implemented are going to be painfully slow
for numbers of this magnitude. We suggest that you download the solution
key for bigint provided to you and use the \texttt{fast\_multiply} and
\texttt{fast\_divide} methods implemented by a certain wonderful TA.

\paragraph{Milestone Requirements:}\label{milestone-requirements-1}

Your program will receive an unknown (greater than 2) number of
command-line arguments. These will be the names of the training files
and finally the name of the test file. You are to calculate the
\textbf{trigram frequency vector} for each of the files given. You will
then compute the \textbf{cosine similarity} between the frequency vector
of the test file and the frequency vector of each training file. You
will find the highest value for similarity and then print the name of
the language that was the best match (followed by a new-line character).

You must name your output file \textbf{language.o}.

\textbf{Sample command-line input:}

\texttt{\$ ./language.o english german spanish icelandic maori test}

If \texttt{icelandic} happened to be the best match for \texttt{test},
your command-line should end up looking like:

\texttt{\$ ./language.o english german spanish icelandic maori test}\\
\texttt{icelandic}\\
\texttt{\$ }

\paragraph{Grading Rubric}\label{grading-rubric-1}

\begin{itemize}
\tightlist
\item
  Functional Correctness: 70\%
\item
  Design, Representation, and Comments: 30\%
\end{itemize}

\textbf{Important Note:} You are not being graded on how fast your
program runs, with one caveat: if your program is so slow or so
memory-inefficient that Mimir cannot run it, you will receive no points
for functional correctness. A well-written program should run in
anywhere from 5-60 seconds on the data set we have provided. If your
program takes multiple minutes to run on your conputer, then something
is wrong.

\subsection{Getting Started}\label{getting-started}

We provide (via GitHub) a large data-set in addition to this document
and some started code. You can obtain it via:

\texttt{\$ git clone https://github.com/csc211/final-project}

You will notice that inside the resulting directory are two
sub-directories called \texttt{training\_languages} and
\texttt{testing\_languages}. Each sub-directory contains some text
files, each named for a language. The text in the training files is
different from the text in the testing files.

\subsection{Helpful Hints}\label{helpful-hints}

\subsubsection{Dealing with the data
files}\label{dealing-with-the-data-files}

If you are anything like some of your instructors, you hate typing
repetitive and unnecessary things on the command line. You should learn
to rely on \emph{wildcards}, also known as \emph{file globbing}. For
example, if you ran:

\texttt{\$ ./language.o training_languages/* testing_languages/english}

the command line will expand the * into \emph{all the files in the
\textbf{training\_languages} sub-directory}, so your program will see
every file individually just as if you had painstakingly typed out the
name of every file individually!

\subsubsection{Writing your own compile
script}\label{writing-your-own-compile-script}

For milestones 1 and 2, you need to write your own compile script. We
have provided you something to start you off but you will need to edit
it. In particular, you need your compile script to take \textbf{every
.cpp file} you have written and compile them into a \textbf{single
executable file}.

Note that ease-of-debugging and speed-of-execution are at odds with one
another. When you are just developing, you \textbf{should not} have
flags like \texttt{-O2} (which stands for optimization level 2), and you
\textbf{should} have a flag like \texttt{-g} (which keeps some symbol
information so your debugger can show you your own source code).
However, prior to submitting your project, you should remove the
\texttt{-g} flag. You can also experiment with putting in the
\texttt{-O2} od \texttt{-O3} flags for compiler optimization. You may
even want to see which optimization flags produce the fastest
executables. Sometimes, the difference can be significant!

You can put a stopwatch to your program with the \texttt{time} command:

\texttt{\$ time ./language.o training_languages/* testing_languages/english}

On my (rather fast) laptop, I can say something like this:

\texttt{\$ time ./language.o training_languages/* testing_languages/english}
\texttt{training_languages/english}


\texttt{real     0m2.610s}\\
\texttt{user     0m2.586s}\\
\texttt{sys      0m0.024s}

The relevant number here is the \texttt{user} time, since the computer
is also busy doing some other stuff (like running my operating system).
Essentially, my program ran in 2.586 seconds.

\subsubsection{Memory Efficiency}\label{memory-efficiency}

If you are not careful, it is easy to make your program use too much
memory or run very slowly (the two are often related). Remember that
arguments are passed to functions by \textbf{value} in C++ by default.
This means that if you take a vector containing \textasciitilde{}20,000
8-byte integers and pass it around by value, C\texttt{++} is making several
copies of that vector. Each copy is worth \textasciitilde{}160 kB. These
can add up very quickly, particularly given that the total size of the
language text files is \textasciitilde{}200 MB. Use the information
presented in class on passing arguments by reference to cut down on this
wasted memory usage.

\section{Good Luck!}\label{good-luck}

\end{document}
